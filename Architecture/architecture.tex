\documentclass[12pt,a4paper,fullpage]{article}
\usepackage[utf8]{inputenc}
\usepackage[russian]{babel}
\usepackage{fullpage}
\usepackage{hyperref}
\usepackage{graphicx}
\usepackage{placeins}
\usepackage{tabulary}
\usepackage{float}

\usepackage{tikz} %for images and graphics
\usetikzlibrary{shapes, positioning, arrows}

%defining img insert func
\newcommand{\pics}[3]{
	\begin{figure}[htbp!]
		\noindent 
		\includegraphics[scale=#1]{./Pics/#2}
		\caption{#3}
	\end{figure}
	\FloatBarrier}

%definig new tikz primitives
\tikzstyle{startstop} = [ellipse, minimum width=3cm, minimum height=1cm,text centered, text width = 3.5cm, draw=black]

\tikzstyle{io} = [trapezium, trapezium left angle=60, trapezium right angle=120, minimum width=3cm, minimum height=1cm, text centered, text width = 3.5cm, draw=black]

\tikzstyle{process} = [rectangle, minimum width=3cm, minimum height=1cm, text centered, text width = 6cm, draw=black]

\tikzstyle{decision} = [diamond, aspect=4, text centered, draw=black]

\tikzstyle{arrow} = [thick,->,>=stealth]

% % % % % % % % % % % % % % % % % % % % % % % % % % % % % % % % % % % % %
\begin{document}

\begin{titlepage}
	
	\begin{center}
		\Huge{Лестницы}
	\end{center}

	\vspace{40pt}
	
	\begin{center}
		\LARGE{Архитектура приложения}
	\end{center}
	
\end{titlepage}

% % % % % % % % % % % % % % % % % % % % % % % % % % % % % % % % % % % % %
\tableofcontents

% % % % % % % % % % % % % % % % % % % % % % % % % % % % % % % % % % % % %
\newpage
\section{Выдержки из технического задания}

\paragraph{Геймплей:} нединамичный платформер.

\paragraph{Цель игры:} выбраться наружу, набрав как можно больше очков. 

\paragraph{Очки} начисляются за:
\begin{itemize}
	\item собранные монетки;
	\item дополнительные жизни;
	\item оставшееся после прохождения уровня время (1 секунда = 1 очко).
\end{itemize}

\paragraph{Уровень:}
\begin{itemize}
	\item три типа (отличаются только оформлением):
	\begin{itemize}
		\item пещера;
		\item подземелье;
		\item поверхность.
	\end{itemize}
	\item содержит от 6 до 36 комнат;
	\item при переходе на следующий:
	\begin{itemize}
		\item происходит по нажатию клавиши (после вывода соответствующего приглашения по обнулении таймера);
		\item собранные жизни сохраняются.
	\end{itemize}
\end{itemize}

\paragraph{Комната:}
\begin{itemize}
	\item занимает окно игры целиком;
	\item соединяется с другими посредством переходов (из любой комнаты есть путь в любую другую);
	\item содержит от 1 до 3 этажей;
	\item этажи соединяются лестницами;
	\item дополнительные элементы:
	\begin{itemize}
		\item перегородки;
		\item двери.
	\end{itemize}
	\item переход из комнаты возможнен с помощью:
	\begin{itemize}
		\item двери в стене комнаты (левая/правая);
		\item прохода сквозь пол/потолок.
	\end{itemize}
\end{itemize}

\paragraph{Персонаж:}
\begin{itemize}
	\item жизни:
	\begin{itemize}
		\item две в начале уровня;
		\item дополнительные могут быть собраны по мере прохождения игры;
		\item после утраты на 3 секунды включается неуязвимость, позволяющая уйти с шипов. Игра завершается, если дополнительных жизней не осталось.
	\end{itemize}
	\item гибель наступает в результате:
	\begin{itemize}
		\item падения с большой высоты (более 2/3 общей высоты комнаты);
		\item соприкосновения с шипами.
	\end{itemize}
	\item передвижение:
	\begin{itemize}
		\item неподвижность;
		\item бег;
		\item прыжок:
		\begin{itemize}
			\item высота --- не более вертикального сегмента;
			\item длина --- не более горзионтального сегмента.
		\end{itemize}
		\item падение:
		\begin{itemize}
			\item происходит, если персонаж добегает до конца платформы и не останавливается.
		\end{itemize}
		\item подъем по лестнице.
		\item сбор предметов происходит автоматически;
		\item двери заблаговременно открываются при приближении.
	\end{itemize}
\end{itemize}

\paragraph{Карта:} 
\begin{itemize}
	\item находится в правом верхнем углу (в свёрнутом состоянии);
	\item показывает посещенные, текущую и стартовую комнаты;
	\item размер --- шесть комнат по вертикали и шесть по горизонтали;
	\item переходы показываются, если существуют.
\end{itemize}

\paragraph{Завершение игры} происхоит в случае:
\begin{itemize}
	\item утраты всех жизней;
	\item по прохождении последнего уровня.
\end{itemize}

\paragraph{Запись лучшего результата} происходит, если:
\begin{itemize}
	\item в таблице рекордов есть пустые места;
	\item текущее значение лучше, чем одно из занесенных в таблицу.
\end{itemize}

\paragraph{Управление} осуществляется с клавиатуры или геймпада.\\
Команды и дефолтные биндинги:\\

\begin{tabulary}{\linewidth}{|C|C|C|}
	\hline
	Команда & Клавиатура & Геймпад\\
	\hline
	Бег влево & A & Крестовина Влево \\
	\hline
	Бег вправо & D & Крестовина Вправо \\
	\hline
	Прыжок & Space & A \\
	\hline
	Меню & Escape & Start \\
	\hline
	Забраться на лестинцу & W & Крестовина Вверх \\
	\hline
	Открыть дверь & E & X \\
	\hline
	Упасть с лестницы & S & Крестовина Вниз \\
	\hline
	Показать карту & M & Y \\
	\hline
	Прокрутить камеру & стрелки & правый джойстик \\
	\hline
\end{tabulary}
\vspace{10pt}

Нажатие <<Esc>> во время игры --- постановка игры на паузу и вызов меню с предложениям продолжить или завершить текущую сессию.\\

\paragraph{Интерфейс:}
\begin{itemize}
	\item символы ``>'' и ``<'' вокруг пункта главного меню означают, что он сейчас выбран, и по нажатии на <<Enter>> (или А) будет выполнено соответствующее действие:
	\begin{itemize}
		\item продолжить --- присутствует, если в прошлом запуске была незаконченная сессия. Иначе отсутствует
		\item старт --- начало игры
		\item рекорды --- таблица рекордов с очками и именами победителей
		\item упрaвление --- меню настроек управления
		\item выход -- выход из приложения\\
	\end{itemize}
	\item игра:	
	\begin{itemize}
		\item основное окно --- текущий уровень;
		\item левый верхний угол --- таймер и счётчик жизней. 
		\item правый верхний угол --- количество очков и полупрозрачный оверлей с мини-картой.
	\end{itemize}
\end{itemize}

\paragraph{Графика:}
\begin{itemize}
	\item OpenGL;
	\item плоская;
	\item пиксельная;
	\item спрайтовая;
	\item анимация предусмотрена для:
	\begin{itemize}
		\item дверей (открытие);
		\item монеток (вращение);
		\item жизней (вращение);
		\item персонажа (движение).
	\end{itemize}
	\item Размер комнаты фиксирован (1440х900 пикселей);
	\item Размер окна:
	\begin{itemize}
		\item можно менять произвольным образом;
		\item минимальный --- 480х300.
	\end{itemize}
\end{itemize}

\paragraph{Физика} обычная платформерная:
\begin{itemize}
	\item сквозь закрытые двери и стены ходить нельзя;
	\item сквозь перекрытия проваливаться и пропрыгивать нельзя;
	\item управлять персонажем в воздухе (влево-вправо) возможно.
\end{itemize}

\paragraph{Настройки:}
\begin{itemize}
	\item хранятся в файле.
\end{itemize}

\paragraph{Лучшие результаты:}
\begin{itemize}
	\item хранятся в файле.
\end{itemize}

\paragraph{Звук и музыка:}
\begin{itemize}
	\item на каждом уровне своя музыка;
	\item каждое действие персонажа во время игры сопровождается звуками;
	\item в меню озвучиваются:
	\begin{itemize}
		\item переключения между пунктами;
		\item подтверждение выбора.
	\end{itemize}
\end{itemize}

\paragraph{Логгирование:}
\begin{itemize}
	\item четыре уровня:
	\begin{itemize}
		\item debug;
		\item info;
		\item warning;
		\item error.
	\end{itemize}
	\item все пишутся в stdout;
	\item пример записи <<LEVEL:[timestamp] Log message>>.
\end{itemize}

\paragraph{<<Ошибки>>:}
\begin{itemize}
	\item Отсутствие файла настроек/ошибка при его чтении:
	\begin{enumerate}
		\item вывести в лог информационное сообщение (в случае ошибки при чтении --- редупреждение) об отсутствии файла и том, что он будет создан и заполнен значениями по умолчанию
		\item создать файл и заполнить значениями по умолчанию
	\end{enumerate}
	\item Отсутствие файла высоких очков:
	\begin{enumerate}
		\item вывести в лог информационное сообщение об отсутствии файла и том, что он будет создан.
		\item создать пустой файл высоких очков
	\end{enumerate}
	\item Ошибка при чтении файла:
	\begin{enumerate}
		\item вывести в лог предупреждение о том, что файл не может быть прочитан, и поэтому будет перезаписан
		\item сохранить старый файл с дополнением ``.corrupted<timestamp>'' к имени файла, где <timestamp> --- текущие время и дата
		\item создать пустой файл
	\end{enumerate}
	\item Отсутствие файла музыки или звука:
	\begin{enumerate}
		\item вывести в лог предупреждение об отсутствующем файле.
		\item в ситуации, когда звук должнен быть воспроизведён, не воспроизводить звук, и не генерировать ошибок.
	\end{enumerate}
	\item Отсутствие файла изображения любого из игровых объектов, или файла шрифта:
	\begin{enumerate}
		\item вывести в лог сообщение об ошибке, с указанием имени отсутствующего файла
		\item завершить работу программы
	\end{enumerate}
\end{itemize}

\subsection{Список вопросов}
\begin{itemize}
	\item что происходит с персонажем поистичение таймера, если уровень не закончен;
	\item начисляются ли очки за оставшиеся "базовые жизни";
	\item конечно ли (каково) число возможных дополнительных жизней.
\end{itemize}


% % % % % % % % % % % % % % % % % % % % % % % % % % % % % % % % % % % % %
\newpage
\section{Архитектура}
На основании проведенного анализа технического задания, можно заключить, что приложение имеет 2 относительно независимых уровня:
\begin{itemize}	
	\item пользовательский;
	\item системный.
\end{itemize}

Взаимодействие между уровнями осуществляется посредством передачи тех или иных комманд и/или сигналов. 

\begin{figure}[thbp!]
	\centering
	\tikzstyle{abstract}=[rectangle, draw=black, rounded corners, fill=blue!3, text centered]
	\begin{tikzpicture}[node distance=1cm, auto, ->, >=stealth']
	%nodes
	\node[abstract, text width = 4cm](usr_level) {Пользовательский уровень};

	\node[abstract, text width = 4cm, below  = of usr_level](sys_level) {Системный уровень};
	
	%edges    
	\path (usr_level.west) edge [bend right = 105] node [left] {Запрос} (sys_level.west);
	\path (sys_level.west) edge [bend left = 35] node [right] {Ответ} (usr_level.west);
	
	\path (sys_level.east) edge [bend right = 105] node [right] {Ответ} (usr_level.east);
	\path (usr_level.east) edge [bend left = 35] node [left] {Запрос} (sys_level.east);
	
	\end{tikzpicture}
	\caption{Схема взаимодействия уровней}
\end{figure}


Рассмотрим каждый из уровней отдельно.

\subsection{Пользовательский уровень}

Функции уровня:
\begin{itemize}
	\item ;
	\item .
\end{itemize}

\subsection{Системный уровень}

Функции уровня:
\begin{itemize}
	\item ;
	\item .
\end{itemize}

\subsection{Совмещенная схема}

% % % % % % % % % % % % % % % % % % % % % % % % % % % % % % % % % % % % %
\newpage
\section{Старая часть}
\subsection{Модули}

Можно выделить следующие логические части программы (делее --- модули), отвечающие за:
\begin{itemize}
	\item Звук.
	\item Графика.
	\item Логика.
	\item Интерфейс пользователя.
	\item Статистика.
	\item Ошибки.
	\item .\\
\end{itemize}

\subsection{Функции модулей}
\subsubsection{Звук}
\begin{itemize}
	\item Проигрывание определенных звуков по вызовам из других модулей.
	\item Переключение звуковых схем.
	\item .\\
\end{itemize}

\subsubsection{Графика}
\begin{itemize}
	\item Отображение соответствующих графических схем.
	\item .\\
\end{itemize}

\subsubsection{Логика}
\begin{itemize}
	\item Текущие вычисления (урон и прочее).
	\item Генераторы.
	\item .\\
\end{itemize}

\subsubsection{Интерфейс пользователя}
\begin{itemize}
	\item Хранение пользовательских настроек.
	\item Передача соответствующих сигналов в другие блоки.
	\item .\\
\end{itemize}

\subsubsection{Статистика}
\begin{itemize}
	\item Хранение и обработка текущих данных (таблицы очков и прочее).
	\item .\\
\end{itemize}

\subsubsection{Ошибки}
\begin{itemize}
	\item .\\
\end{itemize}


\subsection{Взаимодействие модулей}
Взаимодействие модулей может быть проиллюстрировано следующим образом:\\

\begin{figure}[thbp!]
  \centering
  \tikzstyle{abstract}=[rectangle, draw=black, rounded corners, fill=blue!3, text centered]
  \begin{tikzpicture}[node distance=1cm, auto, ->, >=stealth']
  %nodes
    \node[abstract, text width = 4cm](Sound) {Звук};
    \node[abstract, text width = 4cm, right = of Sound](Logic) {Логика};
    \node[abstract, text width = 4cm, below  = of Sound](Graph) {Графика};
    \node[abstract, text width = 4cm, right = of Graph](Stat) {Статистика};
    \node[abstract, text width = 4cm, below  = of Graph](UI) {Интерфейс пользователя};
    \node[abstract, text width = 4cm, right = of UI](Mistakes) {Ошибки};
    
  %edges    
	\path 
		(Logic) edge (Stat)
				edge (Sound)
				edge [bend left = 5] (Graph);
	\path 
		(Graph) edge [bend left = 5] (Logic);
	\path 
		(Stat) edge (Graph);
	\path 
		(UI) edge [bend left = 90] (Sound)
			 edge [bend left = 90] (Graph);
    
  \end{tikzpicture}
  \caption{Схема взаимодействия модулей}
\end{figure}

% % % % % % % % % % % % % % % % % % % % % % % % % % % % % % % % % % % % %
\newpage
\subsection{Информационные потоки}
\begin{figure}[thbp!]
	\begin{tikzpicture}[node distance = 2cm]
	% Nodes
	\node (launch) [startstop] {Старт};
	\node (sys-chk) [process, below of=launch] {Системные проверки};
	\node (game-exist) [decision, below of=sys-chk, yshift=-0.5cm] {Старая игра существует?};
	\node (game-gen) [process, right of=game-exist, xshift=7cm] {Сгенерировать новую};
	\node (game-load) [process, below of=game-exist, yshift=-0.5cm] {Загрузить игру};
	\node (game-launch) [process, below of=game-load] {Запустить игру};
	\node (game-play) [process, below of=game-launch] {Играть в игру};

%	\node (exit) [startstop, below of=game] {выходной сигнал};
	
	% Edges
	\draw [arrow] (launch) -- (sys-chk);
	\draw [arrow] (sys-chk) -- (game-exist);
	\draw [arrow] (game-exist) -- node[anchor=south] {нет} (game-gen);
	\draw [arrow] (game-exist) -- node[anchor=west] {да} (game-load);
	\draw [arrow] (game-gen) |- (game-load);
	\draw [arrow] (game-load) -- (game-launch);
	\draw [arrow] (game-launch) -- (game-play);

	
	\end{tikzpicture}
	\vspace{10pt}
	\caption{Общая схема игры (упрощенная)}
\end{figure}



\subsection{Процессы начала}
\begin{figure}[thbp!]
	\begin{tikzpicture}[node distance = 2cm]
		% Nodes
		\node[startstop](launch) {Старт};
		\node[process, below of = launch](sys-chk) {Проверка наличия системных компонентов};
		\node[decision, below of = sys-chk, yshift=-0.5cm](sys-chk-cond) {Проверка пройдена успешно?};
		\node[process, right of = sys-chk-cond, xshift = 7cm](sys-rep) {Механизм исправления ошибок};
		\node[process, below of = sys-chk-cond, yshift=-0.5cm](usr-chk) {Проверка наличия пользовательских компонентов};
		\node[decision, below of = usr-chk, yshift=-0.5cm](usr-chk-cond) {Проверка пройдена успешно?};
		\node[process, right of = usr-chk-cond, xshift = 7cm](usr-rep) {Механизм исправления ошибок};
		\node[process, below of = usr-chk-cond, yshift=-0.5cm](game) {Запуск игры};
		
		% Edges
		\draw [arrow] (launch) -- (sys-chk);
		\draw [arrow] (sys-chk) -- (sys-chk-cond);
		\draw [arrow] (sys-chk-cond) -- node[anchor=south] {нет} (sys-rep);
		\draw [arrow] (sys-chk-cond) -- node[anchor=west] {да} (usr-chk);
		\draw [arrow] (sys-rep) |- (sys-chk);
		\draw [arrow] (usr-chk) -- (usr-chk-cond);
		\draw [arrow] (usr-chk-cond) -- node[anchor=south] {нет} (usr-rep);
		\draw [arrow] (usr-chk-cond) -- node[anchor=west] {да} (game);		
		\draw [arrow] (usr-rep) |- (usr-chk);
		
	\end{tikzpicture}
	\vspace{10pt}
	\caption{Группа процессов начала}
\end{figure}

\subsection{Процессы игры}
\begin{figure}[thbp!]
	\centering
	\begin{tikzpicture}[node distance = 2cm, auto]
	
	\end{tikzpicture}
	\vspace{10pt}
	\caption{Группа процессов игры}
\end{figure}

\subsection{Процессы завершения}
\begin{figure}[thbp!]
	\centering
	\begin{tikzpicture}[node distance = 2cm, auto]
	
	\end{tikzpicture}
	\vspace{10pt}
	\caption{Группа процессов завершения}
\end{figure}

% % % % % % % % % % % % % % % % % % % % % % % % % % % % % % % % % % % % %
\end{document}