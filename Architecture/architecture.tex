\documentclass[12pt,a4paper,fullpage]{article}
\usepackage[utf8]{inputenc}
\usepackage[russian]{babel}
\usepackage{fullpage}
\usepackage{hyperref}
\usepackage{graphicx}
\usepackage{placeins}
\usepackage{float}

\usepackage{tikz} %for images and graphics
\usetikzlibrary{positioning, arrows}

\begin{document}

\tableofcontents
\newpage

\section{Модули}
Можно выделить следующие логические части программы (делее --- модули), отвечающие за:
\begin{itemize}
	\item Звук.
	\item Графика.
	\item Логика.
	\item Интерфейс пользователя.
	\item Статистика.
	\item Ошибки.
	\item .\\
\end{itemize}

\subsection{Функции модулей}
\subsubsection{Звук}
\begin{itemize}
	\item Проигрывание определенных звуков по вызовам из других модулей.
	\item Переключение звуковых схем.
	\item .\\
\end{itemize}

\subsubsection{Графика}
\begin{itemize}
	\item Отображение соответствующих графических схем.
	\item .\\
\end{itemize}

\subsubsection{Логика}
\begin{itemize}
	\item Текущие вычисления (урон и прочее).
	\item Генераторы.
	\item .\\
\end{itemize}

\subsubsection{Интерфейс пользователя}
\begin{itemize}
	\item Хранение пользовательских настроек.
	\item Передача соответствующих сигналов в другие блоки.
	\item .\\
\end{itemize}

\subsubsection{Статистика}
\begin{itemize}
	\item Хранение и обработка текущих данных (таблицы очков и прочее).
	\item .\\
\end{itemize}

\subsubsection{Ошибки}
\begin{itemize}
	\item .\\
\end{itemize}


\subsection{Взаимодействие модулей}
Взаимодействие модулей может быть проиллюстрировано следующим образом:\\

\begin{figure}[thbp!]
  \centering
  \tikzstyle{abstract}=[rectangle, draw=black, rounded corners, fill=blue!3, text centered]
  \begin{tikzpicture}[node distance=1cm, auto, ->, >=stealth']
  
  %nodes
    \node[abstract, text width = 4cm](Sound) {Звук};
    \node[abstract, text width = 4cm, right = of Sound](Logic) {Логика};
    \node[abstract, text width = 4cm, below  = of Sound](Graph) {Графика};
    \node[abstract, text width = 4cm, right = of Graph](Stat) {Статистика};
    \node[abstract, text width = 4cm, below  = of Graph](UI) {Интерфейс пользователя};
    \node[abstract, text width = 4cm, right = of UI](Mistakes) {Ошибки};
    
  %edges    
	\path 
		(Logic) edge (Stat)
				edge (Sound)
				edge [bend left = 5] (Graph);
	\path 
		(Graph) edge [bend left = 5] (Logic);
	\path 
		(Stat) edge (Graph);
	\path 
		(UI) edge [bend left = 90] (Sound)
			 edge [bend left = 90] (Graph);
    
  \end{tikzpicture}
  \caption{Схема взаимодействия модулей}
\end{figure}


\end{document}