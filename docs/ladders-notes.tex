\documentclass[12pt,a4paper,fullpage]{article}
\usepackage[utf8]{inputenc}
\usepackage[russian]{babel}
\usepackage{fullpage}
\usepackage{hyperref}
\usepackage{graphicx}
\usepackage{float}

\begin{document}

\newcommand{\pic}[2] {
  %\begin{figure}[h]
  \includegraphics[scale=#1]{./Pics/#2}}
%\end{figure}

\section{Название:} Лестницы.
\section{Геймплей:} Нединамичный платформер.

Цель игры --- не погибнув, как можно быстрее дойти до последнего уровня и выбраться наружу, набрав как можно больше очков. Очки начисляются за собранные монетки и дополнительные жизни, время, оставшееся от таймера, стартующего в начале прохождения уровня.

Каждый уровень представляет собой несколько комнат, каждая комната занимает целиком окно игры. Комнаты соединены переходами, в каждой из них --- от одного до трёх этажей. Могут также быть вертикальные перегородки, двери. Этажи могут быть соединены лестницами.

Персонаж может погибнуть, либо упав с большой высоты, либо наступив на шипы. Изначально даётся две запасных жизни. В процессе прохождения можно собирать дополнительные.

Также можно собирать монетки, за что начисляются дополнительные очки.

В игре три уровня:
\begin{enumerate}
\item Пещера,
\item Подземелье
\item Поверхность
\end{enumerate}
В каждом уровне от шести до тридцати шести комнат. Уровни отличаются только оформлением.

Потерять жизнь можно, упав с большой высоты или наступив на шипы.

\section{Интерфейс:}
\subsection{Главное меню}

\noindent \pic{0.27}{mainmenu}
Символы ``>'' и ``<'' вокруг пункта означают, что он сейчас выбран, и по нажатии на энтер (или А) будет выполнено соответствующее действие.

\begin{itemize}
\item Продолжить --- присутствует, если в прошлом запуске была незаконченная сессия. Иначе отсутствует
\item Старт --- начало игры
\item Рекорды --- таблица рекордов с очками и именами победителей
\item Управление --- меню настроек управления
\item Выход -- выход из приложения
\end{itemize}
\subsection{Игра}
\noindent \pic{0.27}{interface}

Основное окно --- текущий уровень. В левом верхнем углу таймер и счётчик жизней. В правом верхнем углу количество очков и полупрозрачный оверлей с мини-картой.
\subsection{Карта и мини-карта}
Начальная и текущая комнаты помечаются отдельными цветами. Переходы между комнатами также наносятся на карту.
Большая карта появляется при нажатии определённой кнопки. На ней можно рассмотреть не только переходы между комнатами, но и интерьер уже посещённых комнат.\\
\pic{0.27}{map-big.png}
\section{Игровая механика:}
\subsection{Уровни}
Уровень представляет собой несколько (от 6 до 36) комнат, причём из любой комнаты есть путь в любую другую комнату, то есть уровень представляет собой \href{http://ru.wikipedia.org/wiki/%D0%9E%D1%80%D0%B8%D0%B5%D0%BD%D1%82%D0%B8%D1%80%D0%BE%D0%B2%D0%B0%D0%BD%D0%BD%D1%8B%D0%B9_%D0%B3%D1%80%D0%B0%D1%84#.D0.A1.D0.B2.D1.8F.D0.B7.D0.BD.D0.BE.D1.81.D1.82.D1.8C
}{\textit{сильно связный орграф}}.

Переход из комнаты в комнату возможен тремя способами:
\begin{itemize}
\item Выйти через дверь в левой или правой стене комнаты
\item Пролезть по лестнице сквозь пол или потолок
\item Провалиться сквозь отсутствующий пол первого этажа комнаты
\end{itemize}

\subsection{Комнаты}
Каждая комната состоит из девяти равных \textit{областей} (разделена по вертикали и горизонтали на три части). Три соседние области, расположенные по горизонтали, образуют \textit{этаж}. В каждой комнате три этажа, каждый из которых, впрочем, может отсутствовать.
Таким образом, в каждой комнате есть по 12 вертикальных и горизонтальных \textit{сегментов стен}, каждый из которых может присутствовать или отсутствовать. Также, в них могут располагаться двери и лестницы.\\
\pic{0.27}{room.png}
Если отсутствует внешний сегмент, это значит, что у комнаты со стороны отсутствующего сегмента также отсутствует соответствующий сегмент --- чтобы получится изотропный проход или пролёт. Со стороны, с которой к комнате не примыкает соседняя комната, внешние сегменты отсутствовать не могут.
Область считается присутствующей, если сегмент пола присутствует.
Внутри комнаты, из области в область можно попасть:
\begin{itemize}
\item перейдя на соответствующий сегмент
\item пройдя сквозь дверь или отсутствующую вертикальную перегородку
\item спустившись или поднявшись по лестнице
\item \textit{спрыгнув} на находящийся ниже этаж, только на один. То есть можно спрыгнуть:
  \begin{itemize}
  \item с 3-го этажа на 2-й
  \item со 2-го этажа на 1-й
  \item с 1-го на 3-й находящейся внизу комнаты
  \end{itemize}
\end{itemize}

Комнату также можно рассматривать как граф. Области являются вершинами, а двери, лестницы и спрыги --- рёбрами. Граф всех областей всех комнат также должен быть сильно связным.\\
\pic{0.27}{room-example-1}
\noindent \pic{0.27}{room-example-2}
\noindent \pic{0.27}{room-example-3}
\noindent \pic{0.27}{room-example-4}

Конфигурация комнат такова, чтобы выполнялось требование о сильной связности графа уровня. Из этого следует, в частности, что комнаты могут быть разделены на части, недостижимые одна из другой, но в таком случае путь будет лежать через другие комнаты.
\subsection{Карта}
В свёрнутом состоянии тусуется в правом верхнем углу, показывает посещённые комнаты, текущую комнату и стартовую комнату. Размер карты --- шесть комнат по вертикали и шесть по горизонтали. У комнат показываются переходы, если они есть. Стартовая комната отмечена тёмно-зелёным, текущая --- голубым. Цвета не принципиальны, на самом деле, и должны храниться в настройках. Вполне вероятно, что такая цветовая гамма будет хреново смотреться и восприниматься.\\
\pic{1.0}{map.png}

В развернутом (по нажатию кнопки) состоянии таймер игры встаёт на паузу, карта занимает весь экран, на ней становится виден интерьер комнат (показан только интерьер текущей комнаты, потому что я рисовать всё это вручную \textit{утомительно}. Но для остальных комнат он тоже типа есть):\\
\pic{0.27}{map-big.png}

\subsection{Передвижение персонажа}
Персонаж может находиться в пяти состояниях:
\begin{itemize}
\item Стоит
\item Бежит
\item Прыгает
\item Падает
\item Забирается по лестнице
\end{itemize}
Монетки и жизни подбираются автоматически. Двери заблаговременно открываются при приближении персонажа.
\subsubsection{Граф переходов состояний}
Переходы из состояния в состояние возможны по следующей схеме:\\
\pic{1.0}{character-moves.png}

\subsubsection{Бездействие}
Персонаж просто стоит. Может начать бежать, взбираться по лестнице, или прыгнуть
\subsubsection{Прыжок}
Предназначен для перепрыгивания шипов и отсутствующих сегментов, а также для спрыгивания.

Высота подпрыгивания --- не больше вертикального сегмента, то есть запрыгнуть с нижнего этажа на верхний нельзя, --- только забраться по лестнице.

Общая длина прыжка (вместе с последующим падением) --- более горизонтального сегмента, чтобы можно было перепрыгнуть отсутствующий горизонтальный сегмент.
\subsubsection{Падение}
Падение происходит при достижении верхней точки прыжка, а также в том случае, если персонаж добегает до конца платформы и не останавливается.

Если высота падения превышает определённую (два вертикальных сегмента), то отнимается одна жизнь.
\subsubsection{Движение по лестнице}
Если персонаж находится достаточно близко к лестнице, он может начать двигаться по ней. Вверх или вниз. С лестницы можно сойти, в таком случае персонаж просто переходит в состояние ``Падение''. Можно с лестницы спрыгнуть --- перейти в состояние ``Прыжок'' и набрать высоту перед падением и приземлением.

Преодолеть часть лестницы, проходящую сквозь сегмент, можно двумя способами:
\begin{itemize}
\item Вверх --- двигаясь по лестнице, или находясь в состоянии ``Прыжок'' --- когда персонаж долез до отверстия и прыгнул в него.
\item Вниз --- двигаясь по лестнице, или находясь в состоянии ``Падение''. То есть, на отверстии лестницы можно стоять, и через него можно перебегать, не опасаясь провалиться. Однако, для ускорения движения по уровню можно заранее прыгнуть и пролететь сквозь отверстие, через которое проходит лестница.
\end{itemize}
\subsection{Жизни}
Потерять жизнь можно, упав с большой высоты или наступив на шипы. Большая высота -- более двух сегментов. То есть при падении с третьего этажа до первого или ниже. Считается только вертикальный перепад высоты, поэтому если подпрыгнуть максимально высоко (и далеко) и приземлиться на этаж, находящийся сразу под тем, с которого был совершён прыжок, жизнь не потратится.

После утраты жизни на 3 секунды включается неуязвимость, позволяющая уйти с шипов. Если дополнительных жизней не остаётся, игра завершается.
\subsection{Переход на следующий уровень}
При достижении игроком выхода с уровня, оставшееся (если оно осталось) время превращается в очки, из расчёт, например, 1 секунда --- 1 очко.

Таймер начинает убывать со скоростью 5 секунд за секунду, очки соответственно прибавляться. При обнулении таймера выводится приглашение нажать на любую кнопку для перехода на следующий уровень. После нажатия кнопки загружается следующий уровень, таймер сбрасывается на начальное время. Собранные жизни переходят на следующий уровень.
\subsection{Завершение игры}
В случае утраты всех жизней, или если игрок довёл персонажа до выхода из последнего уровня, игра завершается. Если результат достоин занесения в таблицу рекордов (то есть, если в таблице рекордов есть пустые места, ИЛИ если результат лучше, чем один из внесённых в таблицу), игроку показывается сообщение об этом с предложением ввести своё имя.
\section{Управление:}
С клавиатуры или геймпада.
Команды и дефолтные биндинги:\\
\begin{tabular}{|p{6cm}|p{2.5cm}|p{3cm}|}
\hline
Команда & Клавиатура & Геймпад\\
\hline
Бег влево & A & Крестовина Влево \\
\hline
Бег вправо & D & Крестовина Вправо \\
\hline
Прыжок & Space & A \\
\hline
Меню & Escape & Start \\
\hline
Забраться на лестницу & W & Крестовина Вверх \\
\hline
Открыть дверь & E & X \\
\hline
Упасть с лестницы & S & Крестовина Вниз \\
\hline
Показать карту & M & Y \\
\hline
Прокрутить камеру & стрелки & правый джойстик \\
\hline
\end{tabular}

При прокрутке камеры она возвращается в исходное положение после отпускания кнопки прокрутки.

При нажатии Esc во время игры --- игра ставится на паузу и возникает меню с предложениями продолжить или завершить текущую игру.
\section{Графика:}
OpenGL. Плоская. Пиксельная. Спрайтовая. Принцип работы спрайтовой анимации понятен из вот этой картинки:\\
\pic{1}{utsuho.png}
Анимация предусмотрена для дверей, монеток и жизней (вращение), и персонажа.

Размер каждой комнаты фиксирован --- 1440х900 пикселей. Размер окна можно менять произвольным образом. Если весь уровень не влезает в окно, включается прокрутка камеры, центрованной на персонаже.

Минимальный размер окна --- 480х300. Меньше этого размера окно уменьшить нельзя. При изменении размеров окна соответственно меняется положение интерфейса.
\section{Физика:}
Обычная платформерная. Сквозь закрытые двери и стены ходить нельзя, сквозь перекрытия проваливаться и пропрыгивать нельзя. Управлять персонажем в воздухе (влево-вправо) возможно.
\section{Настройки:}
Хранятся в файле.
\section{Лучшие результаты:}
Хранятся в файле.
\section{Звук и музыка:}
Присутствуют. В идеале --- на каждом уровне своя музыка, каждое действие персонажа во время игры сопровождается звуками.

В меню озвучиваются переключения между пунктами и подтверждение выбора.
\section{Логирование:}
Три уровня логов --- debug, info, warning и error, они все пишутся в stdout. Формат логов не принципиален, но например может быть <<LEVEL:[timestamp] Log message>>.
\section{<<Ошибки>>:}
Предусмотренные граничные состояния (связанные с ресурсами) и реакции на них:
\begin{itemize}
\item Отсутствие файла настроек либо ошибка при его чтении
  \begin{enumerate}
  \item вывести в лог информационное сообщение (в случае ошибки при чтении --- предупреждение) об отсутствии файла и том, что он будет создан и заполнен значениями по умолчанию
  \item создать файл и заполнить значениями по умолчанию
  \end{enumerate}
\item Отсутствие файла высоких очков:
  \begin{enumerate}
  \item вывести в лог информационное сообщение об отсутствии файла и том, что он будет создан.
  \item создать пустой файл высоких очков
  \end{enumerate}
  Либо ошибка при чтении файла:
  \begin{enumerate}
  \item вывести в лог предупреждение о том, что файл не может быть прочитан, и поэтому будет перезаписан
  \item сохранить старый файл с дополнением ``.corrupted<timestamp>'' к имени файла, где <timestamp> --- текущие время и дата
  \item создать пустой файл
  \end{enumerate}
\item Отсутствие файла музыки или звука:
  \begin{enumerate}
  \item вывести в лог предупреждение об отсутствующем файле.
  \item в ситуации, когда звук должен быть воспроизведён, не воспроизводить звук, и не генерировать ошибок.
  \end{enumerate}
\item Отсутствие файла изображения любого из игровых объектов, или файла шрифта
  \begin{enumerate}
  \item вывести в лог сообщение об ошибке, с указанием имени отсутствующего файла
  \item завершить работу программы
  \end{enumerate}
\end{itemize}
\end{document}
