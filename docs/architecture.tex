\documentclass[12pt,a4paper,fullpage,titlepage]{article}
\usepackage[utf8]{inputenc}
\usepackage[russian]{babel}
\usepackage{fullpage}
\usepackage{hyperref}
\usepackage{graphicx}
\usepackage{placeins}
\usepackage{tabularx}

\usepackage{tikz} %for images and graphics
\usetikzlibrary{shapes, positioning, arrows}

%definig new tikz primitives connection shemes
\tikzstyle{abstract}=[rectangle, draw=black, rounded corners, fill=blue!3, text centered, text width=4cm]

\tikzstyle{level1}=[rectangle, draw=black, rounded corners, fill=blue!3, text centered, text width=4cm]

\tikzstyle{level2}=[rectangle, draw=black, rounded corners, fill=green!3, minimum height=4cm, minimum width=6cm, text centered, text width=5cm, text depth=3cm]

\tikzstyle{level3}=[rectangle, draw=black, rounded corners, fill=red!3, minimum height=8cm, minimum width=14cm, text centered, text width=13cm, text depth=7cm]

%definig new tikz primitives for workflows
\tikzstyle{startstop} = [ellipse, minimum width=5cm, minimum height=1cm, text centered, text width=4cm, draw=black]

\tikzstyle{io} = [trapezium, trapezium left angle=60, trapezium right angle=120, minimum width=3cm, minimum height=1cm, text centered, text width=3.5cm, draw=black]

\tikzstyle{process} = [rectangle, minimum width=3cm, minimum height=1cm, text centered, text width=6cm, draw=black]

\tikzstyle{decision} = [diamond, aspect=4, text centered, draw=black]

\tikzstyle{arrow} = [thick,->,>=stealth]

% % % % % % % % % % % % % % % % % % % % % % % % % % % % % % % % % % % % %
\begin{document}

\begin{titlepage}

	\begin{center}
		\Huge{Лестницы}
	\end{center}

	\vspace{40pt}

	\begin{center}
		\LARGE{Архитектура приложения}
	\end{center}

\end{titlepage}

% % % % % % % % % % % % % % % % % % % % % % % % % % % % % % % % % % % % %
\tableofcontents

% % % % % % % % % % % % % % % % % % % % % % % % % % % % % % % % % % % % %
\newpage
\listoftables

% % % % % % % % % % % % % % % % % % % % % % % % % % % % % % % % % % % % %
%\newpage
\listoffigures

% % % % % % % % % % % % % % % % % % % % % % % % % % % % % % % % % % % % %
\newpage
\section{Выдержки из технического задания}

\paragraph{Геймплей:} нединамичный платформер.

\paragraph{Цель игры:} выбраться наружу, набрав как можно больше очков.

\paragraph{Очки} начисляются за:
\begin{itemize}
	\item собранные монетки;
	\item дополнительные жизни;
	\item оставшееся после прохождения уровня время (1 секунда = 1 очко).
\end{itemize}

\paragraph{Уровень:}
\begin{itemize}
	\item три типа (отличаются только оформлением):
	\begin{itemize}
		\item пещера;
		\item подземелье;
		\item поверхность.
	\end{itemize}
	\item содержит от 6 до 36 комнат;
	\item при переходе на следующий:
	\begin{itemize}
		\item происходит по нажатию клавиши (после вывода соответствующего приглашения по обнулении таймера);
		\item собранные жизни сохраняются.
	\end{itemize}
\end{itemize}

\paragraph{Комната:}
\begin{itemize}
	\item занимает окно игры целиком;
	\item соединяется с другими посредством переходов (из любой комнаты есть путь в любую другую);
	\item содержит от 1 до 3 этажей;
	\item этажи соединяются лестницами;
	\item дополнительные элементы:
	\begin{itemize}
		\item перегородки;
		\item двери.
	\end{itemize}
	\item переход из комнаты возможен с помощью:
	\begin{itemize}
		\item двери в стене комнаты (левая/правая);
		\item прохода сквозь пол/потолок.
	\end{itemize}
\end{itemize}

\paragraph{Персонаж:}
\begin{itemize}
	\item жизни:
	\begin{itemize}
		\item две в начале уровня;
		\item дополнительные могут быть собраны по мере прохождения игры;
		\item после утраты на 3 секунды включается неуязвимость, позволяющая уйти с шипов. Игра завершается, если дополнительных жизней не осталось.
	\end{itemize}
	\item гибель наступает в результате:
	\begin{itemize}
		\item падения с большой высоты (более 2/3 общей высоты комнаты);
		\item соприкосновения с шипами.
	\end{itemize}
	\item передвижение:
	\begin{itemize}
		\item неподвижность;
		\item бег;
		\item прыжок:
		\begin{itemize}
			\item высота --- не более вертикального сегмента;
			\item длина --- не более горизонтального сегмента.
		\end{itemize}
		\item падение:
		\begin{itemize}
			\item происходит, если персонаж добегает до конца платформы и не останавливается.
		\end{itemize}
		\item подъем по лестнице.
		\item сбор предметов происходит автоматически;
		\item двери заблаговременно открываются при приближении.
	\end{itemize}
\end{itemize}

\paragraph{Карта:}
\begin{itemize}
	\item находится в правом верхнем углу (в свёрнутом состоянии);
	\item показывает посещенные, текущую и стартовую комнаты;
	\item размер --- шесть комнат по вертикали и шесть по горизонтали;
	\item переходы показываются, если существуют.
\end{itemize}

\paragraph{Завершение игры} происходит в случае:
\begin{itemize}
	\item утраты всех жизней;
	\item по прохождении последнего уровня.
\end{itemize}

\paragraph{Запись лучшего результата} происходит, если:
\begin{itemize}
	\item в таблице рекордов есть пустые места;
	\item текущее значение лучше, чем одно из занесенных в таблицу.
\end{itemize}

\paragraph{Управление} осуществляется с клавиатуры или геймпада.\\
\begin{table}[h]
\begin{tabularx}{\textwidth}{|X|c|c|}
	\hline
	Команда & Клавиатура & Геймпад\\
	\hline
	Бег влево & A & Крестовина Влево \\
	\hline
	Бег вправо & D & Крестовина Вправо \\
	\hline
	Прыжок & Space & A \\
	\hline
	Меню & Escape & Start \\
	\hline
	Забраться на лестницу & W & Крестовина Вверх \\
	\hline
	Открыть дверь & E & X \\
	\hline
	Упасть с лестницы & S & Крестовина Вниз \\
	\hline
	Показать карту & M & Y \\
	\hline
	Прокрутить камеру & стрелки & правый джойстик \\
	\hline
\end{tabularx}
\caption{Команды и дефолтные биндинги.}
\end{table}

Нажатие <<Esc>> во время игры --- постановка игры на паузу и вызов меню с предложениям продолжить или завершить текущую сессию.\\

\paragraph{Интерфейс:}
\begin{itemize}
	\item символы ``>'' и ``<'' вокруг пункта главного меню означают, что он сейчас выбран, и по нажатии на <<Enter>> (или А) будет выполнено соответствующее действие:
	\begin{itemize}
		\item продолжить --- присутствует, если в прошлом запуске была незаконченная сессия. Иначе отсутствует
		\item старт --- начало игры
		\item рекорды --- таблица рекордов с очками и именами победителей
		\item управление --- меню настроек управления
		\item выход -- выход из приложения\\
	\end{itemize}
	\item игра:
	\begin{itemize}
		\item основное окно --- текущий уровень;
		\item левый верхний угол --- таймер и счётчик жизней.
		\item правый верхний угол --- количество очков и полупрозрачный оверлей с мини-картой.
	\end{itemize}
\end{itemize}

\paragraph{Графика:}
\begin{itemize}
	\item OpenGL;
	\item плоская;
	\item пиксельная;
	\item спрайтовая;
	\item анимация предусмотрена для:
	\begin{itemize}
		\item дверей (открытие);
		\item монеток (вращение);
		\item жизней (вращение);
		\item персонажа (движение).
	\end{itemize}
	\item Размер комнаты фиксирован (1440х900 пикселей);
	\item Размер окна:
	\begin{itemize}
		\item можно менять произвольным образом;
		\item минимальный --- 480х300.
	\end{itemize}
\end{itemize}

\paragraph{Физика} обычная платформерная:
\begin{itemize}
	\item сквозь закрытые двери и стены ходить нельзя;
	\item сквозь перекрытия проваливаться и пропрыгивать нельзя;
	\item управлять персонажем в воздухе (влево-вправо) возможно.
\end{itemize}

\paragraph{Настройки:}
\begin{itemize}
	\item хранятся в файле.
\end{itemize}

\paragraph{Лучшие результаты:}
\begin{itemize}
	\item хранятся в файле.
\end{itemize}

\paragraph{Звук и музыка:}
\begin{itemize}
	\item на каждом уровне своя музыка;
	\item каждое действие персонажа во время игры сопровождается звуками;
	\item в меню озвучиваются:
	\begin{itemize}
		\item переключения между пунктами;
		\item подтверждение выбора.
	\end{itemize}
\end{itemize}

\paragraph{Логирование:}
\begin{itemize}
	\item четыре уровня:
	\begin{itemize}
		\item debug;
		\item info;
		\item warning;
		\item error.
	\end{itemize}
	\item все пишутся в stdout;
	\item пример записи <<LEVEL:[timestamp] Log message>>.
\end{itemize}

\paragraph{<<Ошибки>>:}
\begin{itemize}
	\item Отсутствие файла настроек/ошибка при его чтении:
	\begin{enumerate}
		\item вывести в лог информационное сообщение (в случае ошибки при чтении --- предупреждение) об отсутствии файла и том, что он будет создан и заполнен значениями по умолчанию
		\item создать файл и заполнить значениями по умолчанию
	\end{enumerate}
	\item Отсутствие файла высоких очков:
	\begin{enumerate}
		\item вывести в лог информационное сообщение об отсутствии файла и том, что он будет создан.
		\item создать пустой файл высоких очков
	\end{enumerate}
	\item Ошибка при чтении файла:
	\begin{enumerate}
		\item вывести в лог предупреждение о том, что файл не может быть прочитан, и поэтому будет перезаписан
		\item сохранить старый файл с дополнением ``.corrupted<timestamp>'' к имени файла, где <timestamp> --- текущие время и дата
		\item создать пустой файл
	\end{enumerate}
	\item Отсутствие файла музыки или звука:
	\begin{enumerate}
		\item вывести в лог предупреждение об отсутствующем файле.
		\item в ситуации, когда звук должен быть воспроизведён, не воспроизводить звук, и не генерировать ошибок.
	\end{enumerate}
	\item Отсутствие файла изображения любого из игровых объектов, или файла шрифта:
	\begin{enumerate}
		\item вывести в лог сообщение об ошибке, с указанием имени отсутствующего файла
		\item завершить работу программы
	\end{enumerate}
\end{itemize}

\subsection{Список вопросов}
\begin{itemize}
	\item что происходит с персонажем по истечении таймера, если уровень не закончен;
	\item начисляются ли очки за оставшиеся "базовые жизни";
	\item конечно ли (каково) число возможных дополнительных жизней;
	\item каков механизм генерации жизней и монеток:
	\begin{itemize}
		\item вместе с уровнем/комнатой;
		\item спонтанно.
	\end{itemize}
	\item логи:
	\begin{itemize}
		\item зачем;
		\item механизм генерации:
		\begin{itemize}
			\item что и при каких условиях записывается.
		\end{itemize}
	\end{itemize}
	\item
\end{itemize}


% % % % % % % % % % % % % % % % % % % % % % % % % % % % % % % % % % % % %
\newpage
\section{Архитектура}

\subsection{Компоненты игры}
На основании проведенного анализа технического задания, можно заключить, что приложение состоит из ряда компонентов, отвечающих за:
\begin{itemize}
	\item ведение статистики;
	\item логирование;
	\item отслеживание ошибок;
	\item связь с пользователем;
	\item генерацию объектов окружения;
	\item аудиовизуальное сопровождение игрового процесса;
	\item выполнение необходимых расчетов в соответствии с <<моделью мира>> \footnote{<<Модель мира>> --- физические и логические законы игровой реальности.}.\\
\end{itemize}

Таким образом, можно выделить следующие системные компоненты и функции, ими выполняемые:
\begin{itemize}
	\item интерфейс пользователя:
	\begin{itemize}
		\item обеспечение взаимодействия (общения) между пользователем и системой;
		\item хранение пользовательских настроек;
		\item передача соответствующих управляющих сигналов в другие блоки.
	\end{itemize}
	\item звук:
	\begin{itemize}
		\item проигрывание звуковых схем по соответствующим вызовам;
		\item переключение звуковых схем по запросу пользователя.
	\end{itemize}
	\item графика:
	\begin{itemize}
		\item своевременное отображение изменений в области игры в ходе ее осуществления;
		\item отображение графической схемы в соответствии с пользовательскими настройками.
	\end{itemize}
	\item статистика:
	\begin{itemize}
		\item хранение и обработка текущих игровых данных (таблицы очков).
	\end{itemize}
	\item логирование:
	\begin{itemize}
		\item составление файла логов определенного формата.\\
	\end{itemize}
	\item ошибки:
	\begin{itemize}
		\item отслеживание наличия необходимых компонентов;
		\item решение проблем их отсутствия;
		\item информирование пользователя о внештатных ситуациях и конфликтах.
	\end{itemize}
	\item генераторы:
	\begin{itemize}
		\item составление уровней;
		\item заполнение комнат.
	\end{itemize}
	\item <<модель мира>> (<<физика>> игры):
	\begin{itemize}
		\item выполнение необходимых расчетов повреждений/перемещений персонажа.
	\end{itemize}
\end{itemize}


\begin{figure}[thbp!]
\centering
\begin{tikzpicture}[node distance=0.5cm, auto, ->, >=stealth']
	%nodes
	\node (Sound) [level1] {Звук};
	\node (Graph) [level1, right=of Sound] {Графика};
	\node (UI) [level1, right=of Graph] {Интерфейс пользователя};

	\node (Stat) [level1, below=of Sound] {Статистика};
	\node (Mistakes) [level1, right=of Stat] {Ошибки};
	\node (Logs) [level1, right=of Mistakes] {Логирование};

	\node (Physics) [level1, below=of Stat] {Физика};
	\node (Generators) [level1, right=of Physics] {Генераторы};

\end{tikzpicture}
\caption{Компоненты игры.}
\end{figure}

В свою очередь, компоненты могут быть сгруппированы в более крупные логические элементы --- модули:

\begin{itemize}
	\item Логика игры:
	\begin{itemize}
		\item технические компоненты:
		\begin{itemize}
			\item ошибки;
			\item логирование;
		\end{itemize}
		\item статистика;
		\item генераторы;
		\item физика.
	\end{itemize}
	\item Аудиовизуальное сопровождение игры:
	\begin{itemize}
		\item звук;
		\item графика.
	\end{itemize}
	\item Интерфейс пользователя.
\end{itemize}

\begin{figure}[thbp!]
\centering
\begin{tikzpicture}[node distance=1cm, auto, ->, >=stealth']
	%nodes
	\node (Logic) [level2, minimum height=7cm, text depth=6cm] {Логика игры};
		\node (Mistakes) [level1, yshift=1.5cm] {Ошибки};
		\node (Logs) [level1, below of=Mistakes] {Логирование};
		\node (Stat) [level1, below of=Logs] {Статистика};
		\node (Generators) [level1, below of=Stat] {Генераторы};
		\node (Physics) [level1, below of=Generators] {Физика};

	\node (AV) [level2, right=of Logic, yshift=1.5cm] {Аудиовизуальное сопровождение игры};
		\node (Sound) [level1,right=of Mistakes, xshift=1.5cm] {Звук};
		\node (Graph) [level1, below of=Sound] {Графика};

	\node (UI) [level1, below=of AV, fill=green!3, minimum width=6cm, minimum height=1.5cm, text width=6cm] {Интерфейс пользователя};
5\end{tikzpicture}
\caption{Модули игры.}
\end{figure}

\newpage
Все модули можно разделить на два относительно независимых уровня:
\begin{itemize}
	\item пользовательский;
	\item системный.
\end{itemize}

\begin{figure}[thbp!]
\centering
\begin{tikzpicture}[node distance=1cm, auto, ->, >=stealth']
	%nodes
	\node (Sys) [level3] {Системный уровень};
		\node (Logic) [level2, minimum height=6.5cm, text depth=5cm, xshift=-3.5cm, yshift=-0.5cm] {Логика игры};
			\node (Mistakes) [level1, xshift=-3.5cm, yshift=1cm] {Ошибки};
			\node (Logs) [level1, below of=Mistakes] {Логирование};
			\node (Stat) [level1, below of=Logs] {Статистика};
			\node (Generators) [level1, below of=Stat] {Генераторы};
			\node (Physics) [level1, below of=Generators] {Физика};
		\node (AV) [level2, right=of Logic, yshift=1cm] {Аудиовизуальное сопровождение игры};
			\node (Sound) [level1,right=of Mistakes, xshift=1.7cm, yshift=-0.5cm] {Звук};
			\node (Graph) [level1, below of=Sound] {Графика};

	\node (UI) [level1, below=of Sys, fill=red!3, minimum width=14cm, minimum height=1.5cm, text width=13cm] {Интерфейс пользователя};

\end{tikzpicture}
\caption{Уровни игры.}
\end{figure}

\newpage
\subsubsection{Взаимодействие компонентов}
Предположим, что взаимодействие между уровнями и компонентами осуществляется посредством передачи тех или иных команд и/или сигналов.

\begin{figure}[thbp!]
	\centering
	\begin{tikzpicture}[node distance=1cm, auto, ->, >=stealth']
	%nodes
	\node (usr_level) [abstract] {Пользовательский уровень};

	\node (sys_level) [abstract, below  = of usr_level] {Системный уровень};

	%edges
	\path (usr_level.west) edge [bend right=105] node [left] {Запрос} (sys_level.west);
	\path (sys_level.west) edge [bend left=35] node [right] {Ответ} (usr_level.west);

	\path (sys_level.east) edge [bend right=105] node [right] {Ответ} (usr_level.east);
	\path (usr_level.east) edge [bend left=35] node [left] {Запрос} (sys_level.east);

	\end{tikzpicture}
	\caption{Схема взаимодействия уровней.}
\end{figure}

Для определения перечня необходимых команд и управляющих сигналов и ответных сообщений, рассмотрим процессы, происходящие во время игры с точки зрения информационных потоков.

% % % % % % % % % % % % % % % % % % % % % % % % % % % % % % % % % % % % %
\newpage
\subsection{Информационные потоки}

Начнем анализ информационных потоков с рассмотрения обобщенных схем функционирования системы и будем последовательно уточнять схему, следуя принципу <<от общего к частному>>.

Обычно внутреннее устройство игры выглядит следующим образом:

\begin{figure}[thbp!]
	\centering
	\begin{tikzpicture}[node distance = 1cm]
	\node (MAGIC) [rectangle, draw=black, fill=orange!3, text centered, minimum width=5cm, minimum height=2cm] {MAGIC};

	\end{tikzpicture}
	\vspace{10pt}
	\caption{Обычное представление об игре}
\end{figure}

В простейшем виде игра может быть представлена последовательностью действий пользователя:

\begin{figure}[thbp!]
	\centering
	\begin{tikzpicture}[node distance = 1cm]
	% Nodes
	\node (launch) [process] {Включить игру};
	\node (game) [process, below=of launch] {Играть в игру};
	\node (exit) [process, below=of game] {Выключить игру};

	% Edges
	\draw [arrow] (launch) -- (game);
	\draw [arrow] (game) -- (exit);

	\end{tikzpicture}
	\vspace{10pt}
	\caption{Схема игры с точки зрения пользователя}
\end{figure}

Уже на этом моменте происходит разделение на три относительно независимые группы процессов:
\begin{enumerate}
	\item подготовки игры;
	\item непосредственно игрового процесса;
	\item завершения игры.
\end{enumerate}

При этом можно предположить, что процессы 1 и 3 групп относятся в большей части к системному уровню, а 2 --- к пользовательскому. В дальнейшем обозначим компоненты пользовательского уровня светло-зеленым, а системного --- светло-голубым.

% % % % % % % % % % % % % % % % % % % % % % % % % % % % % % % % % % % % %
\newpage
\subsubsection{Процессы подготовки игры}

\begin{figure}[thbp!]
	\centering
	\begin{tikzpicture}[node distance = 1cm]
	%nodes
	\node (init) [startstop] {Входной сигнал};
	\node (work) [process, below=of init] {Подготовка игры};
	\node (exit) [startstop, below=of work] {Подготовленная игра};

	%edges
	\draw [arrow] (init) -- (work);
	\draw [arrow] (work) -- (exit);

	\end{tikzpicture}
	\vspace{10pt}
	\caption{Начальная схема процессов подготовки игры}
\end{figure}


\begin{figure}[thbp!]
	\centering
	\begin{tikzpicture}[node distance = 1cm]
	%nodes

	%edges

	\end{tikzpicture}
	\vspace{10pt}
	\caption{Итоговая схема процессов подготовки игры}
\end{figure}


% % % % % % % % % % % % % % % % % % % % % % % % % % % % % % % % % % % % %
\newpage
\subsubsection{Игровой процесс}

\begin{figure}[thbp!]
	\centering
	\begin{tikzpicture}[node distance = 1cm]
	%nodes
	\node (init) [startstop] {Подготовленная игра};
	\node (work) [process, below=of init] {Игровой процесс};
	\node (exit) [startstop, below=of work] {Инициация выключения};

	%edges
	\draw [arrow] (init) -- (work);
	\draw [arrow] (work) -- (exit);

	\end{tikzpicture}
	\vspace{10pt}
	\caption{Начальная схема игрового процесса}
\end{figure}

\begin{figure}[thbp!]
	\centering
	\begin{tikzpicture}[node distance = 1cm]
	%nodes

	%edges

	\end{tikzpicture}
	\vspace{10pt}
	\caption{Итоговая схема игрового процесса}
\end{figure}

% % % % % % % % % % % % % % % % % % % % % % % % % % % % % % % % % % % % %
\newpage
\subsubsection{Процессы завершения игры}
\begin{figure}[thbp!]
	\centering
	\begin{tikzpicture}[node distance = 1cm]
	%nodes
	\node (init) [startstop] {Инициация выключения};
	\node (work) [process, below=of init] {Завершение игры};
	\node (exit) [startstop, below=of work] {Выходной сигнал};

	%edges
	\draw [arrow] (init) -- (work);
	\draw [arrow] (work) -- (exit);

	\end{tikzpicture}
	\vspace{10pt}
	\caption{Начальная схема процессов завершения игры}
\end{figure}

\begin{figure}[thbp!]
	\centering
	\begin{tikzpicture}[node distance = 1cm]
	%nodes

	%edges

	\end{tikzpicture}
	\vspace{10pt}
	\caption{Итоговая схема процессов завершения игры}
\end{figure}

% % % % % % % % % % % % % % % % % % % % % % % % % % % % % % % % % % % % %
\newpage
\begin{figure}[thbp!]
	\centering
	\begin{tikzpicture}[node distance = 2cm]
	% Nodes
	\node[startstop](launch) {Старт};
	\node[process, below of = launch](sys-chk) {Проверка наличия системных компонентов};
	\node[decision, below of = sys-chk, yshift=-0.5cm](sys-chk-cond) {Проверка пройдена успешно?};
	\node[process, right of = sys-chk-cond, xshift = 7cm](sys-rep) {Механизм исправления ошибок};
	\node[process, below of = sys-chk-cond, yshift=-0.5cm](usr-chk) {Проверка наличия пользовательских компонентов};
	\node[decision, below of = usr-chk, yshift=-0.5cm](usr-chk-cond) {Проверка пройдена успешно?};
	\node[process, right of = usr-chk-cond, xshift = 7cm](usr-rep) {Механизм исправления ошибок};
	\node[process, below of = usr-chk-cond, yshift=-0.5cm](game) {Запуск игры};

	% Edges
	\draw [arrow] (launch) -- (sys-chk);
	\draw [arrow] (sys-chk) -- (sys-chk-cond);
	\draw [arrow] (sys-chk-cond) -- node[anchor=south] {нет} (sys-rep);
	\draw [arrow] (sys-chk-cond) -- node[anchor=west] {да} (usr-chk);
	\draw [arrow] (sys-rep) |- (sys-chk);
	\draw [arrow] (usr-chk) -- (usr-chk-cond);
	\draw [arrow] (usr-chk-cond) -- node[anchor=south] {нет} (usr-rep);
	\draw [arrow] (usr-chk-cond) -- node[anchor=west] {да} (game);
	\draw [arrow] (usr-rep) |- (usr-chk);

	\end{tikzpicture}
	\vspace{10pt}
	\caption{Группа процессов начала}
\end{figure}



% % % % % % % % % % % % % % % % % % % % % % % % % % % % % % % % % % % % %
\newpage
\subsection{Уровни}
\subsubsection{Системный уровень}

\paragraph{Функции уровня:}
\begin{itemize}
	\item ;
	\item .\\
\end{itemize}

\paragraph{Компоненты уровня:}
\begin{itemize}
	\item ;
	\item .\\
\end{itemize}

\paragraph{Схема уровня:}

\begin{figure}[thbp!]
	\centering

	\caption{Схема системного уровня.}
\end{figure}

% % % % % % % % % % % % % % % % % % % % % % % % % % % % % % % % % % % % %
\newpage
\subsubsection{Пользовательский уровень}

\paragraph{Функции уровня:}
\begin{itemize}
	\item Взаимодействие с пользователем:
	\begin{itemize}
		\item осуществляется через GUI;
		\item ;
		\item ;
		\item ;
		\item .
	\end{itemize}
	\item Передача команд системному уровню.\\
\end{itemize}

\paragraph{Компоненты уровня:}
\begin{itemize}
	\item GUI:
	\begin{itemize}
		\item ;
		\item .
	\end{itemize}
	\item .\\
\end{itemize}

Таким образом, пользовательский уровень является <<управляющим воздействием>> \footnote{В терминологии дисциплины <<Управление проектами>> Григорьева В.К.} для системного уровня.\\

\paragraph{Схема уровня:}
\begin{figure}[thbp!]
	\centering

	\caption{Схема пользовательского уровня.}
\end{figure}

% % % % % % % % % % % % % % % % % % % % % % % % % % % % % % % % % % % % %
\newpage
\subsubsection{Совмещенная схема}
\begin{figure}[thbp!]
	\centering

	\caption{Совмещенная схема модулей.}
\end{figure}

% % % % % % % % % % % % % % % % % % % % % % % % % % % % % % % % % % % % %
\newpage
\subsection{Список вопросов}
\begin{itemize}
	\item
\end{itemize}

% % % % % % % % % % % % % % % % % % % % % % % % % % % % % % % % % % % % %
\newpage
\section{Последовательность разработки}
\subsection{Основные этапы разработки}
\begin{table}[h]
\begin{tabularx}{\textwidth}{|c|X|}
	\hline
	 № & Содержание\\
	\hline
	1 & Создание и согласование Технического Задания\\
	\hline
	2 & Создание и согласование Архитектуры\\
	\hline
	3 & Разработка и Тестирование\\
	\hline
	4 & Релиз\\
	\hline
	 & \\
	\hline
\end{tabularx}
\caption{Основные этапы разработки.}
\end{table}

\subsection{Стадии имплементации}

\begin{table}[h]
\begin{tabularx}{\textwidth}{|c|X|}
	\hline
	№ & Содержание\\
	\hline
	1 & \\
	\hline
	& \\
	\hline
\end{tabularx}
\caption{Стадии имплементации.}
\end{table}

% % % % % % % % % % % % % % % % % % % % % % % % % % % % % % % % % % % % %
\newpage
\section{Общая информация}
\subsection{Команда проекта}
\begin{table}[h]
	\begin{tabularx}{\textwidth}{|l|X|}
		\hline
		Кто & Роль \\
		\hline
		\href{https://github.com/taptap/}{taptap} & формулировка идеи и технического задания\\
		\hline
		\href{https://github.com/attillax}{attillax} & согласование технического задания, разработка архитектуры\\
		\hline
		\href{https://github.com/pankshok}{pankshok} & исправление опечаток =)\\
		\hline
		& \\
		\hline
	\end{tabularx}
	\caption{Команда проекта.}
\end{table}

\subsection{Репозитории проекта}
\begin{table}[h]
	\begin{tabularx}{\textwidth}{|l|X|}
		\hline
		№ & Ссылка \\
		\hline
		1 & \href{https://github.com/taptap/ladders}{основной} \\
		\hline
		& \\
		\hline
	\end{tabularx}
	\caption{Репозитории проекта.}
\end{table}

\subsection{Хронология проекта}
\begin{table}[h]
\begin{tabularx}{\textwidth}{|c|c|c|X|}
	\hline
	№ & Начало & Конец & Что\\
	\hline
	1 & 2015-04-06 & 2015-07-18 & Формулировка и согласование идеи проекта \\
	\hline
	2 & 2015-08-01 & 2015-08-29 & Начало работы над архитектурой проекта \\
	\hline
	3 & 2015-11-12 & 2015-11-12 & Пересмотр концепции архитектуры \\
	\hline
	4 & 2015-11-12 &  & Разработка архитектуры проекта\\
	\hline
	5 &  & 2015-12-22 & Завершение сотрудничества \\
	\hline
	&  &  &  \\
	\hline
\end{tabularx}
\caption{Хронология проекта.}
\end{table}

\newpage
\subsection{Список инструментов}
\begin{table}[h]
	\begin{tabularx}{\textwidth}{|l|X|}
		\hline
		Что & Для чего \\
		\hline
		Linux & Операционная система\\
		\hline
		\TeX & Ведение документации\\
		\hline
		Tikz & Построение диаграмм для \TeX-документации\\
		\hline
		\TeX-Studio & IDE для \TeX \\
		\hline
		Git (via GitHub) & Система контроля версий, осуществление совместной территориально-распрелененной разработки \\
		\hline
		Gimp & Подготовка изображений и прочего графического материала \\
		\hline
		& \\
		\hline
	\end{tabularx}
	\caption{Список инструментов.}
\end{table}


% % % % % % % % % % % % % % % % % % % % % % % % % % % % % % % % % % % % %
\end{document}
