\documentclass[12pt,a4paper]{article}
\usepackage{graphicx}
\usepackage{float}
\usepackage[utf8]{inputenc}
\usepackage[russian]{babel}
\begin{document}

\newcommand{\pic}[2]
           {
%\begin{figure}[h]
             \includegraphics[scale=#1]{#2}
           
%\end{figure}
           }
             
\paragraph{Название:} Лестницы.
\paragraph{Геймплей:} Нединамичный платформер. Каждый уровень представляет собой несколько комнат, каждая комната занимает целиком окно игры. Комнаты соединены переходами, в каждой из них --- от одного до трёх этажей. Могут также быть вертикальные перегородки, двери, бездонные пропасти. Этажи могут быть соединены лестницами. Можно собирать монетки, за что начисляются дополнительные очки. Персонаж может погибнуть, упав либо с большой высоты, либо в бездонную яму. Изначально даётся три жизни. В процессе прохождения можно собирать дополнительные.

В игре три уровня. Пещера, Подземелье и Поверхность. В каждом уровне шесть комнат. Отличаются только оформлением. Цель игры --- не погибнув, дойти до последнего уровня и выбраться наружу, набрав как можно больше очков. Очки начисляются за собранные монетки и время, оставшееся от таймера, стартующего в начале прохождения уровня. Также бонусы за собранные дополнительные жизни.

Потерять жизнь можно, упав с большой высоты или наступив на шипы. Большая высота -- более двух третей высоты комнаты. То есть при падении с третьего этажа до первого.

Подробности истории, логику процесса и прочее оставим за кадром :D
\paragraph{Интерфейс:}
Главное меню --- пять кнопок, по порядку:
\begin{itemize}
\item Продолжить --- присутствует, если в прошлом запуске была незаконченная сессия. Иначе отсутствует
\item Старт
\item Рекорды
\item Упрaвление --- настроки управления
\item Выход
\end{itemize}
Основное окно --- текущий уровень. В левом верхнем углу количество очков и жизней. В правом верхнем углу таймер оставшегося времени и полупрозрачный оверлей с картой.
\paragraph{Игровая механика:}
Уровень представляет собой 6 смежных комнат, причём две смежные комнаты не обязательно соединены, но из любой комнаты есть путь в любую другую комнату. Игрок всегда начинает в одной из самых нижних комнат уровня, а заканчивает в одной из самых верхних.

Игрок начинает с тремя жизнями, нулём очков и неоткрытой картой, на которую нанесёна только текущая комната. По мере продвижения игрока по уровню, новые комнаты наносятся на карту. Текущая комната помечается чем-нибудь. Наличие или отсутсвие дверей между комнатами также отмечается на карте. Интерьер комнат (двери, лестинцы, внутренние стены и перекрытия) также наносится на карту. Карту можно развернуть на весь экран, чтобы рассмотреть поподробнее.
\paragraph{Управление:}
С клавиатуры или геймпада.
Команды и дефолтные биндинги:
\begin{itemize}
\item Бег влево --- A --- Крестовина Влево
\item Бег вправо --- D --- Крестовина Вправо
\item Прыжок --- Space --- A
\item Меню --- Escape --- Start
\item Забраться на лестинцу --- W --- Крестовина Вверх
\item Открыть дверь --- E --- X
\item Упасть с лестницы --- S --- Крестовина Вниз
\item Показать карту --- M --- Y
\item Прокрутить камеру --- стрелки --- правый джойстик
\end{itemize}
При прокрутке камеры она возвращается в исходное положение после отпускания кнопки прокрутки.

При нажатии Esc во время игры --- игра ставится на паузу и возникает меню с предложениями продолжить или завершить текущую игру.
\paragraph{Уровни:}
Генерируются автоматически. Интерьер комнат также генерируется автоматически.
\paragraph{Комнаты:}
Комнаты могут быть разделены на до трёх частей по вертикали и до трёх по горизонтали. Некоторые возможные конфигурации комнат:\\
\pic{0.27}{room-example-1}
\noindent \pic{0.27}{room-example-2}
\noindent \pic{0.27}{room-example-3}
При генерации уровень и комнаты строятся таким образом, чтобы из каждой комнаты была доступна каждая другая. Даже, точнее, любая часть любой комнаты --- можно открыть все двери и пройти по всем лестницам, и всегда вернуться назад, к старту уровня.
\paragraph{Графика:}
OpenGL. Плоская. Пиксельная. Спрайтовая. Принцип работы спрайтовой анимации понятен из вот этой картинки:\\
\pic{1}{utsuho.png}
Анимация предусмотрена для дверей, монеток (вращение), и персонажа.

Размер уровня фиксирован --- 1440х900. Размер окна можно менять произвольным образом. Если весь уровень не влезает в окно, включается прокрутка камеры, центрованной на персонаже.

Минимальный размер окна --- 480х300. Меньше этого размера окно уменьшить нельзя. При изменении размеров окна соответственно меняется положение интерфейса.
\paragraph{Физика:}
Обычная платформерная. Сквозь закрытые двери и стены ходить нельзя, сквозь перекрытия проваливаться и пропрыгивать нельзя. Управлять персонажем в воздухе (влево-вправо) возможно.
\paragraph{Настройки:}
Хранятся в файле.
\paragraph{Лучшие результаты:}
Хранятся в файле.
\paragraph{Звук и музыка:}
Присутствуют. В идеале --- на каждом уровне своя музыка, каждое действие персонажа во время игры сопровождается звуками.

В меню озвучиваются переключения между пунктами и подтверждение выбора.
\end{document}
