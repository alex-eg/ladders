\documentclass[12pt,a4paper]{article}
\usepackage[utf8]{inputenc}
\usepackage[russian]{babel}
\begin{document}
\paragraph{Название:} Лестницы.
\paragraph{Геймплей:} Нединамичный платформер. Каждый уровень представляет собой несколько комнат, каждая комната занимает целиком окно игры. Комнаты соединены переходами, в каждой из них --- от одного до трёх этажей. Могут также быть вертикальные перегородки, двери, бездонные пропасти. Этажи могут быть соединены лестницами. Можно собирать монетки, за что начисляются дополнительные очки. Персонаж может погибнуть, упав либо с большой высоты, либо в бездонную яму. Изначально даётся три жизни. В процессе прохождения можно собирать дополнительные.

В игре три уровня. Пещера, Подземелье и Поверхность. В каждом уровне десять комнат. Отличаются только оформлением. Цель игры --- дойти до последнего уровня и выбраться наружу, набрав как можно больше очков. Очки начисляются за собранные монетки и время, оставшееся от таймера, стартующего в начале прохождения уровня. Также бонусы за собранные дополнительные жизни.

Подробности истории, логику и обоснование и прочее оставим за кадром :D
\paragraph{Интерфейс:}
Главное меню --- пять кнопок:
\begin{itemize}
\item Продолжить (присутствует, если в прошлом запуске была незаконченная сессия)
\item Старт
\item Рекорды
\item Упрвление --- настроки управления
\item Выход
\end{itemize}
Основное окно --- текущий уровень. В левом верхнем углу количество очков и жизней. В правом верхнем углу таймер оставшегося времени и оверлей с картой.
\paragraph{Игровая механика:}
Уровень представляет собой 6 смежных комнат, причём две смежные комнаты не обязательно соединены, но из любой комнаты есть путь в любую другую комнату. Игрок всегда начинает в одной из самых нижних комнат уровня, а заканчивает в одной из самых верхних.

Игрок начинает с тремя жизнями, нулём очков и неоткрытой картой, на которую нанесёна только текущая комната. По мере продвижения игрока по уровню, новые комнаты наносятся на карту. Текущая комната помечается чем-нибудь. Наличие или отсутсвие дверей между комнатами также отмечается на карте. Интерьер комнат также наносится на карту.
\paragraph{Управление:}
WASD, стрелки --- движение. Пробел --- прыжок. В момент нахождения на лестнице персонаж реагирует только на клавиши вверх, вниз и прыжок. В момент нажатия пробела игрок находится на лестинице, он падает вниз, а не подпрыгивает.

При нажатии Esc во время игры --- игра ставится на паузу и возникает меню с предложениями продолжить или завершить текущую игру.
\paragraph{Уровни:}
Генерируются автоматически. Интерьер комнат также генерируется автоматически.
Комнаты могут быть разделены на до трёх частей по вертикали и до трёх по горизонтали. Некоторые возможные конфигурации комнат:

При генерации уровень и комнаты строятся таким образом, чтобы из каждой комнаты была доступна каждая другая. Даже, точнее, любая часть любой комнаты.
\paragraph{Графика:}
Пиксельная. Спрайтовая. Анимация объектов также спрайтовая. Алгоритм примерно следующий:
\end{document}
