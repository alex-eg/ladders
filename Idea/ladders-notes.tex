\documentclass[12pt,a4paper]{article}
\usepackage{graphicx}
\usepackage{float}
\usepackage[utf8]{inputenc}
\usepackage[russian]{babel}
\usepackage[usenames,dvipsnames]{xcolor}

%{\color{red} []}

\begin{document}

\textbf{Цветовые обозначения:}

{\color{blue} [Синий]} --- удаленные элементы

{\color{red} [Красный]} --- возникшие вопросы

{\color{ForestGreen} [Зелёный]} --- ответы на вопросы

\newcommand{\ans}[1]{{\color{ForestGreen}#1}}

\paragraph{Название:} Лестницы.
\paragraph{Геймплей:} Нединамичный платформер. {\color{red} [это как]} \ans{[без стрелялок, врагов, бонусов и волшебных труб, как в марио]}

Каждый уровень представляет собой несколько комнат. Каждая комната занимает целиком окно игры. Комнаты соединены переходами, в каждой из них --- от одного до трёх этажей. Могут также быть вертикальные перегородки, двери, бездонные пропасти. Этажи могут быть соединены лестницами {\color{red} [а могут не быть?]}\ans{[Да, могут не быть]}. {\color{red} [Каким образом тогда попадать на несоединенные этажи?]}

Можно собирать монетки, за что начисляются дополнительные очки. 

Персонаж может погибнуть, упав либо с большой высоты, либо в бездонную яму. Изначально даётся три жизни. В процессе прохождения можно собирать дополнительные.

В игре три уровня.
\begin{itemize}
\item Пещера,
\item Подземелье
\item Поверхность
\end{itemize}

В каждом уровне шесть комнат. Отличаются только оформлением. 

\textit{Цель игры} --- не погибнув, дойти до последнего уровня и выбраться наружу, набрав как можно большее количество очков. Очки начисляются за собранные монетки и время, оставшееся от таймера, стартующего в начале прохождения уровня. Также бонусы {\color{red} [какого рода]}\ans{[N очков за дополнительную жизнь]} за собранные дополнительные жизни.

Потерять жизнь можно, упав с большой высоты или наступив на шипы. Большая высота -- более двух третей высоты комнаты {\color{red} [или этажа]}\ans{[Нет, не этажа. Комнаты]} {\color{red} [т.е. при падении с 1 (2) из 3 герой остается жив, а при 3 из 4 - погибает?]}. То есть при падении с третьего этажа до первого.

Подробности истории, логику процесса и прочее оставим за кадром :D

{\color{red} [возможность сохранения на определенном этапе]}\ans{[При выходе из игры, сохраняется текущий уровень, очки, оставшиеся жизни и время. При загрузке персонаж появляется в начале последней посещённой комнаты]} {\color{red} [При выходе появляется меню "сохранить"? есть ли возможность сохранения промежуточного этапа?]}

\paragraph{Интерфейс:}
Главное меню --- пять кнопок, по порядку {\color{red} [По какому порядку? сверху вниз, слева направо или как]}\ans{[Сверху вниз]}:
\begin{itemize}
\item Продолжить --- присутствует, если в прошлом запуске была незаконченная сессия. Иначе отсутствует
\item Старт
\item Рекорды
\item Упрaвление --- настроки управления
\item Выход
\end{itemize}

{\color{red} [Более развернутое описание функций]}\ans{[Старт --- начало игры. Рекорды --- демонстрация таблицы рекордов. Выход --- выход]}\\

Основное окно --- текущий уровень. В левом верхнем углу количество очков и жизней. В правом верхнем углу таймер оставшегося времени 
{\color{red} [просто обратный отсчет?]}\ans{[Да.]} и полупрозрачный оверлей с картой. {\color{red} [Картой чего]}\ans{[Комнат и всего уровня. См. ниже]}

{\color{red} [Какова область, отображаемая на экране]}\ans{[Не понял вопроса]} {\color{red} [Вся комната, несколько этажей или еще как]}

{\color{red} [Взаимное расположение кнопок и экрана]}\ans{[Не понял вопроса]} {\color{red} [где находятся кнопки относительно основной игровой  области - слева/справа/ и т.д.]}
 
\paragraph{Игровая механика:}
Уровень представляет собой 6 смежных комнат {\color{red} [каким образом? в линию, матцу или как-то еще?]}\ans{[Смежность --- единственное условие. Генерация рандомная, так что каждый раз по-разному.]} {\color{red} [Смежные - имеющие общую стену? может ли стена быть глухой, есть ли возможность прохода через потолок/пол?]}, причём две смежные комнаты не обязательно соединены, но из любой комнаты есть путь в любую другую комнату {\color{red} [как может быть проход, если комнаты не обязательно соединены?]}. Игрок всегда начинает в одной из самых нижних комнат уровня, а заканчивает в одной из самых верхних. 
{\color{red} [Если ли какие-либо законы/обязательные направления обхода комнат?]}
\ans{[Нет. Главное просто дойти до выхода]}\\

Игрок начинает с тремя жизнями {\color{red} [Описание жизней: что они собой представляют, как расходуются, пополняются]}\ans{[Пополняются путём подбирания соответствующих бонусов. Расходуются падением с высоты. Представляют собой бонус в виде сердечка.]}{\color{red} [закон расходования каков? - сразу вся жизнь или каком-то образом по частям]}, нулём очков и неоткрытой картой, на которую нанесёна только текущая комната. По мере продвижения игрока по уровню, новые комнаты наносятся на карту. Текущая комната помечается чем-нибудь. {\color{red} [Чем? Меткой положения игрока на карте?]}\ans{[Подсвечивается, например]} Наличие или отсутсвие дверей между комнатами также отмечается на карте. Интерьер комнат (двери, лестинцы, внутренние стены и перекрытия) также наносится на карту. Карту можно развернуть на весь экран, чтобы рассмотреть поподробнее.{\color{red} [Указываются ли дополнительно комнаты начала и конца игры?]}\ans{[Хороший вопрос. Нет. Только начальная и конечная двери.]}\\

{\color{red} [Механизм набора очков]}
\ans{[По n и m очков за собранную монетку и жизнь. k очков за каждую секунду оставшегося времени]}

{\color{red} [Законы генерации монет и дополнительных жизней]}\ans{[N монет и M жизней на уровень, рандомно, не более k монет в одной комнате и не более j жизней]}

\paragraph{Управление:}
С клавиатуры или геймпада {\color{red} [почему клавиатуры не достаточно]} \ans{[Потому что у меня есть геймпад]}. {\color{red} [Ах ты! а у меня нет, я не знаю, как он работает]}

{\color{blue} [Таблица нагляднее]}

Команды и дефолтные биндинги:\\
\begin{tabular}{|p{10cm}|p{2.5cm}|p{3cm}|}
\hline
Команда & Клавиатура & Геймпад\\
\hline
Бег влево & A & Крестовина Влево \\
\hline
Бег вправо & D & Крестовина Вправо \\
\hline
Прыжок & Space & A \\
\hline
Меню & Escape & Start \\
\hline
Забраться на лестинцу & W & Крестовина Вверх \\
\hline
Открыть дверь & E & X \\
\hline
Упасть {\color{red} [не спрыгнуть?]}\ans{[Без подпрыгивания вверх. Спрыгнуть с лестиницы тоже можно]} с лестницы & S & Крестовина Вниз \\
\hline
Показать карту & M & Y \\
\hline
Прокрутить камеру {\color{red} [Куда и как]}\ans{[В любую из четырёх сторон. См. раздел про графику]} {\color{red} [не понятна именно формулировка - что такое прокрутить камеру]} & стрелки & правый джойстик \\
\hline
\end{tabular}

При прокрутке камеры она возвращается в исходное положение после отпускания кнопки прокрутки. {\color{red} [Механизм работы]}\\

При нажатии Esc во время игры --- игра ставится на паузу и возникает меню с предложениями продолжить или завершить текущую игру.

\paragraph{Уровни:}
Генерируются автоматически. {\color{red} [По каким законам/механизмам]}\ans{[По перечисленным выше, насчёт смежности комнат и доступности любой точки из любой. Конкретный алгоритим предстоит разработать. Думай как о чёрном ящике с определёнными входными и выходными данными]}

Интерьер комнат также генерируется автоматически. {\color{red} [По каким законам/механизмам]}\ans{[Аналогично]}

\paragraph{Комнаты:}
Комнаты могут быть разделены на до трёх частей по вертикали и до трёх по горизонтали{\color{red} [начиная со скольки]}\ans{[С одной]}. Некоторые возможные конфигурации комнат:\\

{\color{blue}[Примеры комнат]}\\

При генерации уровень и комнаты строятся таким образом, чтобы из каждой комнаты была доступна каждая другая {\color{red} [Каким образом]}\ans{[ Перемещением персонажа по уровню]} {\color{red} [имелись ввиду законы доступности - т.е. что откуда и куда ведет, может дополнительные примеры помогут лучше понять. не отдельных комнат, а уровн целиком, всех их и взаимное расположение]}. Даже, точнее, любая часть любой комнаты --- можно открыть все двери и пройти по всем лестницам, и всегда вернуться назад, к старту уровня. {\color{red} [Более подробное/понятное объяснение]}\ans{[Любая часть любой комнаты всегда доступна из любой другой части любой комнаты. Нет балконов, с которых можно спрыгнуть и не забраться обратно]}

\paragraph{Графика:}
OpenGL. Плоская {\color{red} [насколько]}\ans{[2D]}. Пиксельная. Спрайтовая. Принцип работы спрайтовой анимации понятен из вот этой картинки:\\

{\color{blue}[Пример анимации]}\\

Анимация предусмотрена для дверей {\color{red} [открыть/закрыть?]}\ans{[Да]}, монеток (вращение) {\color{red} [законы вращения]}\ans{[Графика спрайтовая, так что вопрос не имеет смысла]}, и персонажа {\color{red} [какие движения]}\ans{[Про персонажа дополню отдельно, нарисовав конечный автомат состояний, и соответствнно анимаций, в том или ином переходе и в том или ином состоянии]}.

Размер уровня фиксирован --- 1440х900 {\color{red} [каких единиц]}.\ans{[Пикселей]} Размер окна можно менять произвольным образом {\color{red} [каким], [что при этом изменяется]}.\ans{[Потянув за уголок. Меняется размер окна]}{\color{red} [меняется ли положение кнопок и прочего окружения при этом, увеличивается и уменьшается только игровая область или "технические области" тоже. т.е. есть ли масштабирование всех элементов сразу]} Если весь уровень не влезает в окно, включается прокрутка камеры {\color{red} [какая]}\ans{[Не понял вопроса]}{\color{red} [что такое прокрутка камеры]}, центрованной {\color{red} [это как]} на персонаже.

Минимальный размер окна --- 480х300 {\color{red} [каких единиц]} \ans{[Пикселей]}. Меньше этого размера окно уменьшить нельзя. При изменении размеров окна соответственно меняется положение интерфейса{\color{red} [По каким механизмам]}\ans{[Остаются на своих местах в углах экрана, не меняя размера и расстояния от края экрана]}.

\paragraph{Физика:}
Обычная платформерная {\color{red} [Описание}\ans{:]}. {\color{red} [ну и где оно?]} Сквозь закрытые двери и стены {\color{red} про}ходить нельзя, сквозь перекрытия проваливаться и пропрыгивать нельзя. Управлять персонажем в воздухе (влево-вправо) возможно.

\paragraph{Настройки:}
Хранятся в файле. {\color{red} [каком]}\ans{[Текстовом. Или бинарном.]}{\color{red} [структура файла]}

\paragraph{Лучшие результаты:}
Хранятся в файле. {\color{red} [каком]}\ans{[Текстовом. Или бинарном.]}{\color{red} [структура файла]}

\paragraph{Звук и музыка:}
Присутствуют. В идеале --- на каждом уровне своя музыка, каждое действие персонажа {\color{red} [совсем каждое?]}\ans{[Да]} во время игры сопровождается звуками.{\color{red} [формат файла]}\ans{[Какого файла?]}

В меню озвучиваются переключения между пунктами и подтверждение выбора.

{\color{red} [Более подробное описание применения]}
\ans{[Применения чего?]} {\color{red} [звуковых схем. как что и когда озвучивается, как регулируется и прочее]}
\end{document}
